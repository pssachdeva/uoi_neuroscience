\documentclass[11pt]{article}

% AMS Packages
\usepackage{amsmath} 
\usepackage{amssymb} 
\usepackage{amsthm}
% Page dimensions
\usepackage[margin=1in]{geometry}
% Images
\usepackage[pdftex]{graphicx} 
% Enumerate package
\usepackage{enumitem} 
\usepackage{array} 
% Fancify pages
\usepackage{fancyhdr} 
% Convert captions on figures to bold font
\usepackage[labelfont=bf,textfont=md]{caption}
% Time New Roman font
\usepackage{times}
% change the line spacing
\usepackage{setspace} 
% SI Units in math type
\usepackage{siunitx}
% extra symbols
\usepackage{textcomp} 
\usepackage{wrapfig}
% tikz 
\usepackage{tikz}
\usetikzlibrary{shapes.geometric, arrows}
\usepackage{bm}
% Change sizes of sections
\usepackage{titlesec}
\titleformat{\section}{\normalfont\large\bfseries}{\thesection}{1em}{}
\titleformat{\subsection}{\normalfont\bfseries}{\thesubsection}{1em}{}
\titleformat{\subsubsection}{\normalfont\small\bfseries}{\thesubsubsection}{1em}{}
% Declare useful math operators
\DeclareMathOperator*{\argmin}{arg\,min}
\DeclareMathOperator*{\plim}{plim}
\DeclareMathOperator{\Tr}{Tr}
\usepackage{float}

\usepackage{algorithm}
\usepackage[noend]{algpseudocode}
\makeatletter
\makeatother

\newcommand{\varendash}[1][5pt]{%
	\makebox[#1]{\leaders\hbox{--}\hfill\kern0pt}%
}

\pagestyle{fancy}
\chead{\textbf{\small Union of Intersections (UoI) for interpretable data driven discovery and prediction in neuroscience}}
\rhead{}
\lhead{}
\headsep 5pt


\begin{document}
\begin{center}
	\textbf{\LARGE Supplement}
\end{center}

\section{Coupling Models}
We applied UoI$_{\text{Lasso}}$ to coupling matrices obtained from the following datasets:
\begin{itemize}
	\item \textbf{(PVC)} Single-unit activity in monkey primary visual cortex (Kohn et al., 2012) in response to drifting gratings.
	\item \textbf{(A1)} Micro-electrocorticography in rat auditory cortex (Bouchard et al., 2018) in response to tone pips.
	\item \textbf{(M1/S1)} Single-unit activity in monkey primary motor and somatorsensory cortices (Makin et al., 2018) during reaches on a grid \textcolor{red}{(in progress)}.
\end{itemize}

\subsection{Monkey Primary Visual Cortex}
The data was segmented into trials based on the stimulus presentation. There were 2400 trials recorded with 3 monkeys, each with 106, 88, and 112 single-units recorded. We treated each response variable as the square-root of the total number of spikes during the trial. 

We applied Lasso and three variants of UoI$_{\text{Lasso}}$ (optimizing for $R^2$, AIC, and BIC) through a 10-fold cross-validation. We compared the models generated by each of these variants to the model obtained via Lasso. Model performance is quantified by selection ratio (fraction of non-zero parameters), explained variance ($R^2$), AIC, and BIC (columns). Models were fit to all three monkeys. The results are shown in Figures \ref{fig:monkey1}, \ref{fig:monkey2}, and \ref{fig:monkey3} below.

\begin{figure}[H]
	\centering
	\scalebox{0.45}{\includegraphics{img/coupling/monkey1_results.pdf}}
	\vspace{-10pt}
	\caption{Monkey 1 model performance: Lasso vs. the three variants of UoI$_{\text{Lasso}}$}
	\label{fig:monkey1}
\end{figure}

\begin{figure}[H]
	\centering
	\scalebox{0.45}{\includegraphics{img/coupling/monkey2_results.pdf}}
	\vspace{-10pt}
	\caption{Monkey 2 model performance: Lasso vs. the three variants of UoI$_{\text{Lasso}}$}
	\label{fig:monkey2}
\end{figure}

\begin{figure}[H]
	\centering
	\scalebox{0.45}{\includegraphics{img/coupling/monkey3_results.pdf}}
	\vspace{-10pt}
	\caption{Monkey 3 model performance: Lasso vs. the three variants of UoI$_{\text{Lasso}}$}
	\label{fig:monkey3}
\end{figure}

\subsection{Rat Auditory Cortex}
The data (from R32-B7) was segmented into trials based on the stimulus presentation. There were 4200 trials recorded from 128 electrodes. We treated each response variable as the peak response of the $z$-scored (to baseline before stimulus presentation) high-gamma analytic amplitude. 

We applied Lasso and three variants of UoI$_{\text{Lasso}}$ (optimizing for $R^2$, AIC, and BIC) through a 10-fold cross-validation. We compared the models generated by each of these variants to the model obtained via Lasso. Model performance is quantified by selection ratio (fraction of non-zero parameters), explained variance ($R^2$), AIC, and BIC (columns). Models were fit to all three monkeys. The results are shown in Figure \ref{fig:ecog} below.

\begin{figure}[H]
	\centering
	\scalebox{0.45}{\includegraphics{img/coupling/ecog_hg_results.pdf}}
	\vspace{-10pt}
	\caption{Micro-electrocorticography model performance: Lasso vs. the three variants of UoI$_{\text{Lasso}}$.}
	\label{fig:ecog}
\end{figure}

\newpage
\section{Retinal data}
We present results from retinal data from the Meister lab. Mice retina were presented with flashing bars which allowed for the calculation of 1D spatio-temporal receptive fields. The receptive fields are shown in the first figure below:

\begin{figure}[H]
	\centering
	\scalebox{0.30}{\includegraphics{img/tuning/retina_strfs.pdf}}
	\caption{Spatio-temporal receptive fields after fitting with different methods.}
\end{figure}

While the UoI fit appears to be contain more non-zero parameters, in reality lasso is less selective:

\begin{figure}[H]
	\centering
	\scalebox{0.45}{\includegraphics{img/tuning/retina_selection_ratio.pdf}}
\end{figure}

Meanwhile, UoI Lasso tends to be more predictive:

\begin{figure}[H]
	\centering
	\scalebox{0.45}{\includegraphics{img/tuning/metrics.pdf}}
\end{figure}

and its sparser distribution of STRF parameters is apparent in histograms:

\begin{figure}[H]
	\centering
	\scalebox{0.35}{\includegraphics{img/tuning/retina_histograms.pdf}}
\end{figure}


\end{document}
