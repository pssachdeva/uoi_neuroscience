\documentclass[11pt]{article}

% AMS Packages
\usepackage{amsmath} 
\usepackage{amssymb} 
\usepackage{amsthm}
% Page dimensions
\usepackage[margin=1in]{geometry}
% Images
\usepackage[pdftex]{graphicx} 
% Enumerate package
\usepackage{enumitem} 
\usepackage{array} 
% Fancify pages
\usepackage{fancyhdr} 
% Convert captions on figures to bold font
\usepackage[labelfont=bf,textfont=md]{caption}
% Time New Roman font
\usepackage{times}
% change the line spacing
\usepackage{setspace} 
% SI Units in math type
\usepackage{siunitx}
% extra symbols
\usepackage{textcomp} 
\usepackage{wrapfig}
% tikz 
\usepackage{tikz}
\usetikzlibrary{shapes.geometric, arrows}
\usepackage{bm}
% Change sizes of sections
\usepackage{titlesec}
\titleformat{\section}{\normalfont\large\bfseries}{\thesection}{1em}{}
\titleformat{\subsection}{\normalfont\bfseries}{\thesubsection}{1em}{}
\titleformat{\subsubsection}{\normalfont\small\bfseries}{\thesubsubsection}{1em}{}
% Declare useful math operators
\DeclareMathOperator*{\argmin}{arg\,min}
\DeclareMathOperator*{\plim}{plim}
\DeclareMathOperator{\Tr}{Tr}
\usepackage{float}

\usepackage{algorithm}
\usepackage[noend]{algpseudocode}
\makeatletter
\makeatother

\newcommand{\varendash}[1][5pt]{%
	\makebox[#1]{\leaders\hbox{--}\hfill\kern0pt}%
}

\pagestyle{fancy}
\chead{\textbf{\small Union of Intersections (UoI) for interpretable data driven discovery and prediction in neuroscience}}
\rhead{}
\lhead{}
\headsep 5pt

% title
\title{\textbf{Union of Intersections (UoI) for for Interpretable Data Driven Discovery and Prediction in Neuroscience}}
\author{\Large Supplementary Information}
\date{}
\begin{document}
\thispagestyle{empty}
\maketitle
\section{Coupling Models}
We applied UoI$_{\text{Lasso}}$ to coupling matrices obtained from the following datasets:
\begin{itemize}
	\item \textbf{(PVC)} Single-unit activity taken from monkey primary visual cortex (Kohn et al., 2012) in response to drifting gratings.
	\item \textbf{(A1)} Micro-electrocorticography recordings taken from rat auditory cortex (Bouchard et al., 2018) in response to tone pips.
	\item \textbf{(M1/S1)} Single-unit activity taken from monkey primary motor and somatorsensory cortices (Makin et al., 2018) during reaches on a grid of targets.
\end{itemize}

Each coupling model attempted to linearly predict the activity of a target neuron (electrode) using the activities of the remaining neurons (electrodes). We fit the coupling model with Lasso and three variants of UoI$_{\text{Lasso}}$ (optimizing for $R^2$, AIC, and BIC) through a 10-fold cross-validation. We compared the models generated by each of these variants to the model obtained via Lasso. We quantified model performance by the selection ratio (fraction of non-zero parameters), explained variance ($R^2$), Akaike Information Criterion, and the Bayesian Information Criterion. The two information criterions are defined as
\begin{align}
	\text{AIC} &= 2k - 2 \log(\hat{L}) \label{eqn:aic}\\
	\text{BIC} &= \log(n) \cdot k - 2 \log(\hat{L}) \label{eqn:bic}
\end{align}
where $\log(\hat{L})$ is the log-likelihood of the model on the test set, $n$ is the number of test samples, and $k$ is the number of features. In contrast to explained variance, lower information criterion is preferable. Thus, AIC and BIC penalize models with larger numbers of parameters (first term in equations \ref{eqn:aic} and \ref{eqn:bic}). In addition, the BIC generally prefers sparser models more strongly than the AIC.
\subsection{PVC}
This dataset was recorded by Matthew Smith and Adam Kohn and obtained through the pvc-11 dataset from CRCNS. Spiking activity was recorded from V1 in three anesthetized macaque monkeys during presentations of drifting gratings. Single-unit spiking responses were segmented into trials according to stimulus presentation. In total, there were 2400 trials recorded with 3 monkeys, each with 106, 88, and 112 single-units, respectively.

We caculated each response variable as the square-root of the spike count in the trial. Thus, the coupling model predicts whether the (square-rooted) spike count in a given trial can be described linearly by the (square-rooted) spike counts according to the remaining neurons in the population. The performance of Lasso compared to the three variants of UoI$_{\text{Lasso}}$ is depicted in Figure \ref{fig:pvc} for each monkey. We observe that UoI$_{\text{Lasso}}$ provides sparser models (top row) at limited or no cost to predictive power (second row) resulting in improved information criterion scores (bottom two rows; recall that lower information criterion is better). In addition, the three variants of UoI$_{\text{Lasso}}$ prefer different levels of sparsity, with $UoI_{\text{Lasso}}-$BIC resulting in the sparsest models.


\begin{figure}[t]
	\centering
	\scalebox{0.24}{\includegraphics{img/coupling/pvc11_monkey1.pdf}}
	\scalebox{0.24}{\includegraphics{img/coupling/pvc11_monkey2.pdf}}
	\scalebox{0.24}{\includegraphics{img/coupling/pvc11_monkey3.pdf}}

	\caption{Lasso ($x$-axis) vs. the three variants of UoI$_{\text{Lasso}}$ ($y$-axes) on Monkey 1 (Left), Monkey 2 (Center), and Monkey 3 (Right). Metrics considered are, in order of rows, selection ratio (number of non-zero parameters divided by total possible number of parameters), explained variance ($R^2$), Akaike Information Criterion, and the Bayesian Information Criterion.}
	\label{fig:pvc}
\end{figure}


\subsection{A1}
\begin{wrapfigure}{r}{0.35\textwidth}
	\vspace{-60pt}
	\centering
	\scalebox{0.26}{\includegraphics{img/coupling/ecog_HG.pdf}}
	\caption{Lasso ($x$-axis) vs. the three variants of UoI$_{\text{Lasso}}$ ($y$-axes) on rat auditory cortex coupling models. Rows are metrics while columns are variants of UoI$_{\text{Lasso}}$.}
	\label{fig:ecog}
\end{wrapfigure}
This dataset was recorded by the Bouchard Lab (the specific rat was R32-B7). Micro-electrocorticography was performed on the surface of the auditory cortex in anesthetized rats during the presentation of tone pips at varying frequencies and attenuations. The $\mu$ECoG recordings were  $z$-scored relative to baseline before stimulus presentation and segmented into trials based on the stimulus type. There were 4200 trials (30 frequencies, 7 attenuations, and 20 repetitions) and 128 electrodes. 

We calculated each response variable as the peak response of the $z$-scored high-gamma analytic amplitude. Thus, the coupling model predicts whether the peak response in a given trial can be described linearly by the peak responses of the remaining electrodes. A comparison of Lasso to the variants of UoI$_{\text{Lasso}}$ are shown in Figure \ref{fig:ecog}. Note that the explained variance plots cover the range $[0.9, 1.0]$, since the coupling models were highly predictive. UoI$_{\text{Lasso}}$ resulted in considerably sparser models coming at little to no cost to explained variance.  Similarly to the previous Figure, we observe that there is a spectrum of enforced sparsity, with UoI$_{\text{Lasso}}$--BIC providing the sparsest models. For this dataset, UoI$_{\text{Lasso}}$--AIC provides a good balance between sparsity and predictive power.

\subsection{M1}
\begin{wrapfigure}{r}{0.35\textwidth}
	\vspace{-25pt}
	\centering
	\scalebox{0.26}{\includegraphics{img/coupling/nhp_indy_20160411_01.pdf}}
	\caption{Lasso ($x$-axis) vs. the three variants of UoI$_{\text{Lasso}}$ ($y$-axes) on nonhuman primate coupling models..}
	\label{fig:nhp}
\end{wrapfigure}
This dataset was recorded by the Sabes Lab (obtained through Zenodo). Single-unit responses were obtained from non-human primate primary motor cortex during self-paced reaches to target on a grid. The primate was prompted to reach for a specific target without gaps or pre-movement delay intervals. We binned the total spike trains for all the recorded single-units to widths of 500 milliseconds. According to this binning, there were 1716 samples for 196 single-units.

We caculated each response variable as the square-root of the spike count in the bin. Thus, the coupling model predicts whether the (square-rooted) spike count in a given bin can be described linearly by the (square-rooted) spike counts according to the remaining neurons in the population. 

The performance of Lasso compared to the three variants of UoI$_{\text{Lasso}}$ is depicted in Figure \ref{fig:nhp} for a given primate (Indy) and session (04-11-2016-1). Interestingly, UoI$_{\text{Lasso}}$ provides sparser models except for a subset of neurons under the UoI$_{\text{Lasso}}-R^2$ variant. However, UoI$_{\text{Lasso}}$ still maintains predictive performance, resulting in simpler but predictive models.


\newpage

\section{Tuning}



\end{document}
