\documentclass[11pt]{article}

% AMS Packages
\usepackage{amsmath} 
\usepackage{amssymb} 
\usepackage{amsthm}
% Page dimensions
\usepackage[margin=1in]{geometry}
% Images
\usepackage[pdftex]{graphicx} 
% Enumerate package
\usepackage{enumitem} 
\usepackage{array} 
% Fancify pages
\usepackage{fancyhdr} 
% Convert captions on figures to bold font
\usepackage[labelfont=bf,textfont=md]{caption}
% Time New Roman font
\usepackage{times}
% change the line spacing
\usepackage{setspace} 
% SI Units in math type
\usepackage{siunitx}
% extra symbols
\usepackage{textcomp} 
\usepackage{wrapfig}
% tikz 
\usepackage{tikz}
\usetikzlibrary{shapes.geometric, arrows}
\usepackage{bm}
% Change sizes of sections
\usepackage{titlesec}
\titleformat{\section}{\normalfont\large\bfseries}{\thesection}{1em}{}
\titleformat{\subsection}{\normalfont\bfseries}{\thesubsection}{1em}{}
\titleformat{\subsubsection}{\normalfont\small\bfseries}{\thesubsubsection}{1em}{}
% Declare useful math operators
\DeclareMathOperator*{\argmin}{arg\,min}
\DeclareMathOperator*{\plim}{plim}
\DeclareMathOperator{\Tr}{Tr}
\usepackage{float}

\usepackage{algorithm}
\usepackage[noend]{algpseudocode}
\makeatletter
\makeatother

\newcommand{\varendash}[1][5pt]{%
	\makebox[#1]{\leaders\hbox{--}\hfill\kern0pt}%
}

\pagestyle{fancy}
\chead{\textbf{\small Union of Intersections (UoI) for interpretable data driven discovery and prediction in neuroscience}}
\rhead{}
\lhead{}
\headsep 5pt

% title
\title{\textbf{Union of Intersections (UoI) for for Interpretable Data Driven Discovery and Prediction in Neuroscience}}
\author{\Large Supplementary Information}
\date{}
\begin{document}
\thispagestyle{empty}
\maketitle
\section{Coupling Models}
We applied UoI$_{\text{Lasso}}$ to coupling matrices obtained from the following datasets:
\begin{itemize}
	\item \textbf{(PVC)} Single-unit activity taken from monkey primary visual cortex (Kohn et al., 2012) in response to drifting gratings.
	\item \textbf{(A1)} Micro-electrocorticography recordings taken from rat auditory cortex (Bouchard et al., 2018) in response to tone pips.
	\item \textbf{(M1/S1)} Single-unit activity taken from monkey primary motor and somatorsensory cortices (Makin et al., 2018) during reaches on a grid of targets.
\end{itemize}

Each coupling model attempted to linearly predict the activity of a target neuron (electrode) using the activities of the remaining neurons (electrodes). We fit the coupling model with Lasso and three variants of UoI$_{\text{Lasso}}$ (optimizing for $R^2$, AIC, and BIC) through a 10-fold cross-validation. We compared the models generated by each of these variants to the model obtained via Lasso. We quantified model performance by the selection ratio (fraction of non-zero parameters), explained variance ($R^2$), Akaike Information Criterion, and the Bayesian Information Criterion. The two information criterions are defined as
\begin{align}
	\text{AIC} &= 2k - 2 \log(\hat{L}) \label{eqn:aic}\\
	\text{BIC} &= \log(n) \cdot k - 2 \log(\hat{L}) \label{eqn:bic}
\end{align}
where $\log(\hat{L})$ is the log-likelihood of the model on the test set, $n$ is the number of test samples, and $k$ is the number of features. In contrast to explained variance, lower information criterion is preferable. Thus, AIC and BIC penalize models with larger numbers of parameters (first term in equations \ref{eqn:aic} and \ref{eqn:bic}). In addition, the BIC generally prefers sparser models more strongly than the AIC.
\subsection{PVC}
This dataset was recorded by Matthew Smith and Adam Kohn and obtained through the pvc-11 dataset from CRCNS. Spiking activity was recorded from V1 in three anesthetized macaque monkeys during presentations of drifting gratings. Single-unit spiking responses were segmented into trials according to stimulus presentation. In total, there were 2400 trials recorded with 3 monkeys, each with 106, 88, and 112 single-units, respectively.

We caculated each response variable as the square-root of the spike count in the trial. Thus, the coupling model predicts whether the (square-rooted) spike count in a given trial can be described linearly by the (square-rooted) spike counts according to the remaining neurons in the population. The performance of Lasso compared to the three variants of UoI$_{\text{Lasso}}$ is depicted in Figure \ref{fig:pvc} for each monkey. We observe that UoI$_{\text{Lasso}}$ provides sparser models (top row) at limited or no cost to predictive power (second row) resulting in improved information criterion scores (bottom two rows; recall that lower information criterion is better). In addition, the three variants of UoI$_{\text{Lasso}}$ prefer different levels of sparsity, with $UoI_{\text{Lasso}}-$BIC resulting in the sparsest models.


\begin{figure}[t]
	\centering
	\scalebox{0.24}{\includegraphics{img/coupling/pvc11_monkey1.pdf}}
	\scalebox{0.24}{\includegraphics{img/coupling/pvc11_monkey2.pdf}}
	\scalebox{0.24}{\includegraphics{img/coupling/pvc11_monkey3.pdf}}

	\caption{Lasso ($x$-axis) vs. the three variants of UoI$_{\text{Lasso}}$ ($y$-axes) on Monkey 1 (Left), Monkey 2 (Center), and Monkey 3 (Right). Metrics considered are, in order of rows, selection ratio (number of non-zero parameters divided by total possible number of parameters), explained variance ($R^2$), Akaike Information Criterion, and the Bayesian Information Criterion.}
	\label{fig:pvc}
\end{figure}


\subsection{A1}
\begin{wrapfigure}{r}{0.35\textwidth}
	\vspace{-60pt}
	\centering
	\scalebox{0.26}{\includegraphics{img/coupling/ecog_HG.pdf}}
	\caption{Lasso ($x$-axis) vs. the three variants of UoI$_{\text{Lasso}}$ ($y$-axes) on rat auditory cortex coupling models. Rows are metrics while columns are variants of UoI$_{\text{Lasso}}$.}
	\label{fig:ecog}
\end{wrapfigure}
This dataset was recorded by the Bouchard Lab (the specific rat was R32-B7). Micro-electrocorticography was performed on the surface of the auditory cortex in anesthetized rats during the presentation of tone pips at varying frequencies and attenuations. The $\mu$ECoG recordings were  $z$-scored relative to baseline before stimulus presentation and segmented into trials based on the stimulus type. There were 4200 trials (30 frequencies, 7 attenuations, and 20 repetitions) and 128 electrodes. 

We calculated each response variable as the peak response of the $z$-scored high-gamma analytic amplitude. Thus, the coupling model predicts whether the peak response in a given trial can be described linearly by the peak responses of the remaining electrodes. A comparison of Lasso to the variants of UoI$_{\text{Lasso}}$ are shown in Figure \ref{fig:ecog}. Note that the explained variance plots cover the range $[0.9, 1.0]$, since the coupling models were highly predictive. UoI$_{\text{Lasso}}$ resulted in considerably sparser models coming at little to no cost to explained variance.  Similarly to the previous Figure, we observe that there is a spectrum of enforced sparsity, with UoI$_{\text{Lasso}}$--BIC providing the sparsest models. For this dataset, UoI$_{\text{Lasso}}$--AIC provides a good balance between sparsity and predictive power.

\subsection{M1}
\begin{wrapfigure}{r}{0.35\textwidth}
	\vspace{-25pt}
	\centering
	\scalebox{0.26}{\includegraphics{img/coupling/nhp_indy_20160411_01.pdf}}
	\caption{Lasso ($x$-axis) vs. the three variants of UoI$_{\text{Lasso}}$ ($y$-axes) on nonhuman primate coupling models.}
	\label{fig:nhp}
\end{wrapfigure}
This dataset was recorded by the Sabes Lab (obtained through Zenodo). Single-unit responses were obtained from non-human primate primary motor cortex during self-paced reaches to target on a grid. The primate was prompted to reach for a specific target without gaps or pre-movement delay intervals. We binned the total spike trains for all the recorded single-units to widths of 500 milliseconds. According to this binning, there were 1716 samples for 196 single-units.

We caculated each response variable as the square-root of the spike count in the bin. Thus, the coupling model predicts whether the (square-rooted) spike count in a given bin can be described linearly by the (square-rooted) spike counts according to the remaining neurons in the population. 

The performance of Lasso compared to the three variants of UoI$_{\text{Lasso}}$ is depicted in Figure \ref{fig:nhp} for a given primate (Indy) and session (04-11-2016-1). Interestingly, UoI$_{\text{Lasso}}$ provides sparser models except for a subset of neurons under the UoI$_{\text{Lasso}}-R^2$ variant. However, UoI$_{\text{Lasso}}$ still maintains predictive performance, resulting in simpler but predictive models.\\

\newpage

\section{Tuning}
\subsection{A1}

\begin{wrapfigure}{r}{0.35\textwidth}
	\vspace{-40pt}
	\centering
	\scalebox{0.26}{\includegraphics{img/tuning/ecog_hg.pdf}}
	\caption{Lasso ($x$-axis) vs. the three variants of UoI$_{\text{Lasso}}$ ($y$-axes) for auditory tuning data.}
	\label{fig:ecog_tuning_scores}
	\vspace{-30pt}
\end{wrapfigure}
We fit tuning curves to the auditory $\mu$ECoG recordings using Gaussian basis functions tiling the log-frequency axis. Specifically, the response $y_i$ for a given electrode was modeled as
\begin{align}
y_i(f) &= a_0 + \sum_{j=1}^k a_j \exp\left(-\frac{(f - f_j)^2}{2\sigma_i^2}\right)
\end{align}
where $f_j$ are spread evenly in log-frequency, $k$ is the number of Gaussians (features), and $\sigma_i$ is the spread of the Gaussian basis functions, generally chosen uniformly across all electrodes. The relevant tuning parameters to be fit are the $a_j$. For the $\mu$-ECoG data, we used 7 Gaussians spread evenly from  0.5 to 32 kHz, and $\sigma_i^2=0.64$ octaves. We fit this model using Lasso and the UoI$_{\text{Lasso}}$ variants.

The model performance is shown in Figure \ref{fig:ecog_tuning_scores}. Note that the maximum number of features that can be used is 7, so differences in selection ratio denote a difference in only a few features. In this case, UoI$_{\text{Lasso}}-R^2$ uses fewer features to achieve the same predictive performance for almost all electrodes. There is little different in the information criteria because the number of features is small compared to the amount of data.

In addition, we also examined tuning curves across the $\mu$ECoG grid. The curves are shown in Figure \ref{fig:ecog_tuning_curves}, plotted against log-frequency. Lasso and UoI$_{\text{Lasso}}$ generally agree on almost all electrodes. Both fitting procedures result in broad tonotopy across the grid. For some electrodes, UoI$_{\text{Lasso}}-$BIC (dashed red) zeros out all coefficients except the intercept. In these cases, the penalty for requiring parameters is greater than the relatively weak model performance (observe the scale in Figure \ref{fig:ecog_tuning_scores}).
\begin{figure}[b!]
	\centering
	\scalebox{0.435}{\includegraphics{img/tuning/ecog_grid.pdf}}

	\caption{Auditory tuning curves plotted against log-frequency for each electrode on the $\mu$ECoG grid. Three separate fits are shown: Lasso (black), UoI$_{\text{Lasso}}-R^2$ (red) and UoI$_{\text{Lasso}}-$BIC (dashed red).}
	\label{fig:ecog_tuning_curves}
\end{figure}

\subsection{Retina}

We fit spatio-temporal receptive fields to retinal data on data obtained by the Meister lab. This stimulus in this dataset consists of random bars (ON/OFF) flashed to isolated retinal ganglion cells from mice. Since the data is effectively one-dimensional, the resulting STRFs will be one-dimensional in space. We calculated STRFs by performing spike-triggered averaging for different delays (sweeping over some predetermined window length). The resulting fitted coefficients can be visualized on a plot with time on one axis and space on the other.

We fit STRFs to 10 different recordings. An example STRF (comparing Lasso to UoI$_{\text{Lasso}}$)  along with statistics across the 10 fits is shown in Figure \ref{fig:strfs} below. We observe that the STRFs generated by UoI$_{\text{Lasso}}$ are much sparser and cleaner. Features that are extraneous for predicting the neural responses are removed when applying UoI$_{\text{Lasso}}$. This is quantified by the selection ratio and change in $R^2$ (specifically, change in $R^2$ from baseline to peak):  Figure \ref{fig:strfs}, right. We note that we used the AIC estimation score, which offered the best balance between choosing models that were predictive but also sparse. While the $R^2$ estimation score also worked well, the resulting STRFs did not look at clean.

\begin{figure}[h!]
	\centering
	\scalebox{1}{\includegraphics{img/tuning/example_strf.pdf}}
	\scalebox{1}{\includegraphics{img/tuning/strf_statistics.pdf}}
	
	\caption{\textbf{Left:} comparison of STRFs obtained by using Lasso and UoI$_{\text{Lasso}}$ to regularize the spike-triggered averaging. \textbf{Right:} Top, selection ratio comparison across 10 fits. Bottom, the change in $R^2$ from baseline to peak over the window length.}
	\label{fig:strfs}
\end{figure}

\newpage 
\section{Classification}
\subsection{Recordings from Basal Ganglia}
We consider a dataset from the Berke lab involving recordings from the basal ganglia of rats. Specifically, these datasets are taken from the  globus pallidus (GP) and substantia nigra pars reticulata (SNr). These regions generally participate in race pathways during decision making. The task that allowed the experimenters to assess these pathways is as follows:
\begin{itemize}
	\item A rat is prompted to enter its head in one of three ports using a visual cue. 
	\item Once the rat has placed its head through the port, it is then prompted, with a tone, to either go to the left port or right port. The frequency of the tone indicates whether it should be left or right (GO condition).
	\item On some fraction of the trials, the direction tone will be followed by a white noise burst, indicating that the rat should remain in the center port (STOP condition).
	\item In some trials, the rat moves before the GO tone occurs, which is considered a pre-tone failure.
\end{itemize}
Thus, there are several decisions that can be predicted based on the neural recordings. Can we:
\begin{itemize}
	\item Predict whether the rat made a pre-tone error according to the neural activity? (432 trials).
	\item Predict whether the trial was a GO or STOP trial based on the neural activity? (326 trials).
	\item Predict whether the trial was a left or right GO trial based on the neural activity? (186 trials).
\end{itemize}

\subsection{Prediction of pre-tone failure}
We used neural activity to predict whether a rat would fail on a trial pre-tone. Specifically, we examined spike counts of the 18 units in GP  and 36 units in SNr in a timespan following the point when the rat enters the center port. On pre-tone failures and successes, the rat typically had different time lengths between the center-in and center-out (Figure \ref{fig:delays}, left). 

\begin{figure}[b]
	\centering
	\includegraphics[scale=0.265]{img/classification/delay_distribution_pretone.pdf}
	\includegraphics[scale=0.265]{img/classification/delay_distribution_lr.pdf}
	\includegraphics[scale=0.265]{img/classification/delay_distribution_sg.pdf}
	\caption{\textbf{Left:} Time between center-in and center-out for pre-tone failures (red) and successes (blue). \textbf{Middle:} Time between side-cue event and center-out on left (black) and right (red) trials. \textbf{Right:} Time between side-cue event and center-out on stop (black) and go (red) trials.}
	\label{fig:delays}
\end{figure}

First, we examined firing rates of the neurons in GP and SNr. We applied a Gaussian kernel to the spike times with sampling rate 500 Hz and variance 0.075 seconds$^2$ in the window $(C_0, C_0 + \Delta t)$. The firing rates for the GP and SNr neurons are depicted in Figures \ref{fig:gp_pretone} and \ref{fig:snr_pretone}. Among both populations, there are clearly neurons that are not very active, neurons that are highly active but not distinguishable between the two trial conditions, and neurons that are very active and distinguishable between the two trial conditions. Thus, this neural population is suitable for decoding.

\begin{figure}[H]
	\centering
	\includegraphics[scale=0.43]{img/classification/gp_pretone_fr.pdf}
	\includegraphics[scale=0.45]{img/classification/gp_pretone_fits.pdf}
	\caption{\textbf{Top:} Firing rates for GP neurons in the 0.5 second window after center-in. Thin curves denote individual trials, while thick curves denote trial averages. Red curves refer to pre-tone failures while black curves refer to pre-tone successes. \textbf{Bottom:} Logistic regression (top) and UoI$_{\text{Logistic}}$ fits on models predicting pre-tone success/failure using spike counts in the $(C_0, C_0+\Delta t)$ window. Location on grid refers to neuron in top of plot. Features selected out are denoted by and `x'.}
	\label{fig:gp_pretone}
\end{figure}

\begin{figure}[H]
	\centering
	\includegraphics[scale=0.34]{img/classification/snr_pretone_fr.pdf}
	\includegraphics[scale=0.35]{img/classification/snr_pretone_fits.pdf}
	\caption{\textbf{Left:} Firing rates for SNr neurons in the 0.5 second window after center-in. Thin curves denote individual trials, while thick curves denote trial averages. Red curves refer to pre-tone failures while black curves refer to pre-tone successes. \textbf{Right:} Logistic regression (top) and UoI$_{\text{Logistic}}$ fits on models predicting pre-tone success/failure using spike counts in the $(C_0, C_0+\Delta t)$ window. Location on grid refers to neuron in top of plot. Features selected out are denoted by and `x'.}
	\label{fig:snr_pretone}
\end{figure}
\begin{figure}[H]
	\centering
	\includegraphics[scale=0.49]{img/classification/pretone_metrics.pdf}
	\caption{Accuracy (left) and selection ratio (right) for GP and SNr populations across the five folds.}
	\label{fig:pretone_metrics}
\end{figure}

We applied Logistic Regression and UoI$_{\text{Logistic}}$ with an accuracy estimation score to predict the trial condition given the spike counts in the window $(C_0, C_0 + \Delta t)$. We performed fits separately for GP and SNr neurons using 5-fold cross-validation. The resulting fits (calculated as a median across folds) are depicted in Figures \ref{fig:gp_pretone} and \ref{fig:snr_pretone} in a grid that matches the firing rate grid. Generally:
\begin{itemize}
	\item UoI$_{\text{Logistic}}$ exhibits much sparser fits.
	\item These sparser fits come at no cost to predictive accuracy.
	\item These fits correspond to neurons that generally observe different firing patterns for the pre-tone failure and pre-tone success conditions.
\end{itemize}

The first two observations are summarized by the selection ratio and accuracy, plotted for the GP and SNr populations in Figure \ref{fig:pretone_metrics}, where we plot the accuracy on the validation sets (5 points) and the selection ratios of the 5 fits. The accuracy is benchmarked to chance, which is roughly 75\%, because three-quarters of the trials are pre-tone successes.

\subsection{Prediction of left-right trials}
Next, we used neural activity to predict whether a rat went left or right on successful GO trials. Specifically, we examined spike counts of the 18 units in GP  and 36 units in SNr in a timespan following the point when the side-cue tone occurs. The time length between side-cue tone events and center-out events were generally the same across left/right GO trials (Figure \ref{fig:delays}, middle). There were more successful left trials (106) than successful right trials (80).

We considered the window $(T_0, T_0 + \Delta t)$ where $T_0$ is the side-cue event time and $\Delta t$ is 0.5 seconds (i.e. on most trials, the rat has left the center port in this window). As before, we plot the firing rates for the GP and SNr neurons in this window (with the same Gaussian kernel) in Figures \ref{fig:gp_posttone} and \ref{fig:snr_posttone}. Furthermore, we fit classifiers in the same fashion as the previous subsection and plot the median coefficients in the same figures. Our conclusions are:
\begin{itemize}
	\item  UoI$_{\text{Logistic}}$ exhibits much sparser fits for GP. These sparser fits come at no cost to predictive accuracy.
	\item UoI$_{\text{Logistic}}$ exhibits the same degree of sparsity for the SNr neurons. The resulting fits imply that two neurons are highly predictive of left/right trials. Specifically, if these neurons fire, the trial is highly likely to be a successful right trial. 
	\item In the SNr fits, UoI$_{\text{Logistic}}$ provides fitted values that are an order of magnitude larger than regularized Logistic regression.
\end{itemize}
In Figure \ref{fig:posttone_metrics}, we observe the above conclusions quantified. Note that for SNr, we obtained nearly 100\% accuracy using only the two fitted neurons.
\begin{figure}[H]
	\centering
	\includegraphics[scale=0.43]{img/classification/gp_posttone_fr.pdf}
	\includegraphics[scale=0.45]{img/classification/gp_posttone_fits.pdf}
	\caption{\textbf{Top:} Firing rates for GP neurons in the 0.5 second window after side-cue. Thin curves denote individual trials, while thick curves denote trial averages. Red curves refer to successful right trials while black curves refer to successful left trials. \textbf{Bottom:} Logistic regression (top) and UoI$_{\text{Logistic}}$ fits on models predicting left/right using spike counts in the $(C_0, C_0+\Delta t)$ window. Location on grid refers to neuron in top of plot. Features selected out are denoted by and `x'.}
	\label{fig:gp_posttone}
\end{figure}

\begin{figure}[H]
	\centering
	\includegraphics[scale=0.34]{img/classification/snr_posttone_fr.pdf}
	\includegraphics[scale=0.35]{img/classification/snr_posttone_fits.pdf}
	\caption{\textbf{Left:} Firing rates for SNr neurons in the 0.5 second window after center-in. Thin curves denote individual trials, while thick curves denote trial averages. Red curves refer to successful right trials while black curves refer to successful left trials. \textbf{Right:} Logistic regression (top) and UoI$_{\text{Logistic}}$ fits on models predicting left/right using spike counts in the $(C_0, C_0+\Delta t)$ window. Location on grid refers to neuron in top of plot. Features selected out are denoted by and `x'.}
	\label{fig:snr_posttone}
\end{figure}
\begin{figure}[H]
	\centering
	\includegraphics[scale=0.49]{img/classification/posttone_metrics.pdf}
	\caption{Accuracy (left) and selection ratio (right) for GP and SNr populations across the five folds.}
	\label{fig:posttone_metrics}
\end{figure}

\subsection{Prediction of stop-go trials}
Lastly, we used neural activity to predict whether a rat completed either successful STOP or GO trials. Specifically, we examined spike counts of the 18 units in GP  and 36 units in SNr in a timespan following the point when the side-cue tone occurs. The time length between side-cue tone events and center-out events were much different across the STOP/GO trials (Figure \ref{fig:delays}, right). This is expected, as the center-out on STOP trials occurs when the trial ends. There were more successful GO trials (186) than successful STOP trials (33).

We considered the window $(T_0, T_0 + \Delta t)$ where $T_0$ is the side-cue event time and $\Delta t$ is 0.5 seconds (on most GO trials, the rat has left the center port in this window). As before, we plot the firing rates for the GP and SNr neurons in this window (with the same Gaussian kernel) in Figures \ref{fig:gp_stopgo} and \ref{fig:snr_stopgo}. Furthermore, we fit classifiers in the same fashion as the previous subsection and plot the median coefficients in the same figures. Our conclusions are:
\begin{itemize}
	\item  UoI$_{\text{Logistic}}$ exhibits much sparser fits. In the case of SNr, the resulting fit may be too sparse.
	\item In the case of GP, these sparser fits come at no cost to predictive accuracy; however, the predictive accuracy is barely better than chance.
	\item Meanwhile, for SNr, the sparse fits do result in noticeably worse prediction accuracy, but better than chance.
\end{itemize}
In Figure \ref{fig:stopgo_metrics}, we observe the above conclusions quantified.

\begin{figure}[H]
	\centering
	\includegraphics[scale=0.43]{img/classification/gp_stopgo_fr.pdf}
	\includegraphics[scale=0.45]{img/classification/gp_stopgo_fits.pdf}
	\caption{\textbf{Top:} Firing rates for GP neurons in the 0.5 second window after side-cue event. Thin curves denote individual trials, while thick curves denote trial averages. Red curves refer to successful STOP trials while black curves refer to successful GO trials. \textbf{Bottom:} Logistic regression (top) and UoI$_{\text{Logistic}}$ fits on models predicting STOP/GO using spike counts in the $(C_0, C_0+\Delta t)$ window. Location on grid refers to neuron in top of plot. Features selected out are denoted by and `x'.}
	\label{fig:gp_stopgo}
\end{figure}

\begin{figure}[H]
	\centering
	\includegraphics[scale=0.34]{img/classification/snr_stopgo_fr.pdf}
	\includegraphics[scale=0.35]{img/classification/snr_stopgo_fits.pdf}
	\caption{\textbf{Left:} Firing rates for SNr neurons in the 0.5 second window after side-cue event. Thin curves denote individual trials, while thick curves denote trial averages. Red curves refer to successful STOP trials while black curves refer to successful GO trials. \textbf{Right:} Logistic regression (top) and UoI$_{\text{Logistic}}$ fits on models predicting STOP/GO using spike counts in the $(C_0, C_0+\Delta t)$ window. Location on grid refers to neuron in top of plot. Features selected out are denoted by and `x'.}
	\label{fig:snr_stopgo}
\end{figure}
\begin{figure}[H]
	\centering
	\includegraphics[scale=0.49]{img/classification/stopgo_metrics.pdf}
	\caption{Accuracy (left) and selection ratio (right) for GP and SNr populations across the five folds.}
	\label{fig:stopgo_metrics}
\end{figure}
\end{document}
